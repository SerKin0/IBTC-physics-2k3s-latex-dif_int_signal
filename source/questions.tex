\section{Контрольные вопросы}

\begin{librarybox}
	1. Выведите формулу \cref{eq:th:8}
\end{librarybox}
Для идеального интегрирующего четырёхполюсника связь выходного и входного сигналов:
\[
u_{\text{вых}}(t) = \frac{1}{\tau_0} \int_{-\infty}^{t} u_{\text{вх}}(t) \, dt .
\]
Представим входной сигнал через спектральную плотность \( S(j\omega) \):
\[
u_{\text{вх}}(t) = \frac{1}{2\pi} \int_{-\infty}^{\infty} S(j\omega) e^{j\omega t} d\omega .
\]
Тогда выходной сигнал:
\[
u_{\text{вых}}(t) = \frac{1}{\tau_0} \cdot \frac{1}{2\pi} \int_{-\infty}^{\infty} S(j\omega) \frac{1}{j\omega} e^{j\omega t} d\omega .
\]
Сравнивая с общим выражением для линейного четырёхполюсника:
\[
u_{\text{вых}}(t) = \frac{1}{2\pi} \int_{-\infty}^{\infty} S(j\omega) K(j\omega) e^{j\omega t} d\omega ,
\]
получаем коэффициент передачи интегрирующей цепи:
\[
K(j\omega) = \frac{1}{\tau_0 j\omega} = \frac{1}{\tau_0 \omega} e^{-j\frac{\pi}{2}} .
\]
Это и есть формула \cref{eq:th:8}.

\begin{librarybox}
	2. Почему нельзя неограниченно увеличивать постоянную времени интегрирующего четырехполюсника и неограниченно уменьшать -- дифференцирующего?
\end{librarybox}
Из условий удовлетворительного интегрирования \( \tau_0 \omega \gg 1 \) и дифференцирования \( \tau_0 \omega \ll 1 \) следует, что:
- Для интегрирующей цепи увеличение \( \tau_0 \) улучшает выполнение условия \( \tau_0 \omega \gg 1 \) на низких частотах, но при слишком больших \( \tau_0 \) коэффициент передачи \( K(\omega) = \frac{1}{\tau_0 \omega} \) становится очень малым, сигнал на выходе затухает, и практическое выделение интеграла затрудняется из-за шумов и ограничений усилителей.
- Для дифференцирующей цепи уменьшение \( \tau_0 \) улучшает выполнение условия \( \tau_0 \omega \ll 1 \) на высоких частотах, но при слишком малых \( \tau_0 \) коэффициент передачи \( K(\omega) = \tau_0 \omega \) также становится малым, выходной сигнал ослабляется, усиливаются высокочастотные шумы, а паразитные ёмкости и индуктивности начинают влиять на работу цепи.

Таким образом, неограниченное изменение \( \tau_0 \) приводит к неприемлемому ослаблению полезного сигнала и ухудшению соотношения сигнал/шум.

\begin{librarybox}
	3. Чем отличается спектры периодических и непериодических сигналов?
\end{librarybox}
Спектр периодического сигнала дискретный и представляется рядом Фурье:
\[
u(t) = \frac{a_0}{2} + \sum_{n=1}^{\infty} A_n \cos(n\Omega t - \theta_n) = \frac{1}{2} \sum_{n=-\infty}^{\infty} \dot{A}_n e^{j n \Omega t},
\]
где комплексная амплитуда \( \dot{A}_n = \frac{2}{T} \int_{-T/2}^{T/2} u(t) e^{-j n \Omega t} dt \), \( \Omega = \frac{2\pi}{T} \).

Спектр непериодического сигнала непрерывный и описывается преобразованием Фурье:
\[
u(t) = \frac{1}{2\pi} \int_{-\infty}^{\infty} S(j\omega) e^{j\omega t} d\omega, \quad S(j\omega) = \int_{-\infty}^{\infty} u(t) e^{-j\omega t} dt .
\]
Здесь \( S(j\omega) \) — спектральная плотность, а график её модуля \( |S(j\omega)| \) образует сплошную кривую (см. \cref{fig:2b}), в отличие от дискретных линий на \cref{fig:2a}.

\begin{librarybox}
	4. Каким условиям должны удовлетворять приборы, подключаемые ко всходу и выходу исследуемых четырехполюсников? 
\end{librarybox}
Приборы должны иметь достаточно высокое входное сопротивление и малое выходное сопротивление, чтобы минимизировать влияние на работу четырёхполюсника. Для осциллографа это означает, что его входное сопротивление \( R_{\text{вх осц}} \) должно быть много больше сопротивления \( R \) цепи (\( R_{\text{вх осц}} \gg R \)), чтобы не шунтировать цепь. Генератор сигналов должен обеспечивать достаточно низкое выходное сопротивление (\( R_{\text{вых ген}} \ll R \)), чтобы напряжение на входе четырёхполюсника не зависело от нагрузки. Также приборы должны работать в полосе частот, соответствующей спектру исследуемого сигнала, чтобы не искажать его.

\begin{librarybox}
	5. Поясните принцип работы схемы, изображенной на \cref{fig:6}.
\end{librarybox}
На рис. 1 изображён двухполупериодный выпрямитель. Переменное напряжение \( u_{\text{вх}} \) подаётся на трансформатор, затем на диодный мост, который пропускает ток только в одном направлении в нагрузке \( R_{\text{н}} \). Конденсатор \( C \) служит для сглаживания пульсаций. Выходное напряжение \( u_{\text{вых}} \) близко к постоянному с небольшой остаточной пульсацией. Осциллограмма показывает, что отрицательные полуволны входного напряжения преобразуются в положительные на выходе.

\begin{librarybox}
	6. Изобразите векторные диаграммы напряжения для четырехполюсников, представленных на \cref{fig:1}. Проследите, как изменяются соотношения между $U_\text{вых}$ и $U_\text{вх}$ при изменении $\tau_0$.
\end{librarybox}
Для четырёхполюсника на \cref{fig:1a} (дифференцирующего) векторная диаграмма: вектор напряжения на резисторе \( U_R \) опережает ток \( I \) на \( 0^\circ \), напряжение на конденсаторе \( U_C \) отстаёт от тока на \( 90^\circ \), входное напряжение \( U_{\text{вх}} = U_R + U_C \). При малых \( \tau_0 \) (\( \tau_0 \omega \ll 1 \)) \( U_C \gg U_R \), поэтому \( U_{\text{вых}} = U_R \approx \tau_0 \frac{dU_{\text{вх}}}{dt} \) мало по сравнению с \( U_{\text{вх}} \). При увеличении \( \tau_0 \) отношение \( U_R / U_{\text{вх}} \) растёт, но условие дифференцирования ухудшается.

Для четырёхполюсника на \cref{fig:1b} (интегрирующего) векторная диаграмма: \( U_{\text{вх}} = U_R + U_C \), выходное напряжение \( U_{\text{вых}} = U_C \). При больших \( \tau_0 \) (\( \tau_0 \omega \gg 1 \)) \( U_R \gg U_C \), поэтому \( U_{\text{вых}} = U_C \approx \frac{1}{\tau_0} \int U_{\text{вх}} dt \) мало. При уменьшении \( \tau_0 \) отношение \( U_C / U_{\text{вх}} \) увеличивается, но условие интегрирования ухудшается.