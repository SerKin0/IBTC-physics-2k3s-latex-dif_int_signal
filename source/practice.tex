\section{Практическая часть}
\subsection{Расчет параметров}
для дифференцирования:
\[\begin{aligned}
	\tau_0 = 10~\text{мкс} = 10^{-5}~\text{с} && R = 10~\text{кОм} = 10^4~\text{Ом}
\end{aligned}\]
\begin{equation} \label{eq:pr:1}
	C = \frac{\tau_0}{R} = \frac{10^{-5}~\text{с}}{10^4~\text{Ом}} = 10^3~\text{пФ}  = 1\ 000~\text{пФ}
\end{equation}

Для интегрирования:
\[\begin{aligned}
	\tau_0 = 5~\text{мс} = 5 \cdot 10^{-3}~\text{с} && R = 500~\text{кОм} = 5 \cdot 10^{5}~\text{Ом}
\end{aligned}\]
\begin{equation} \label{eq:pr:2}
	C = \frac{\tau_0}{R} = \frac{5 \cdot 10^{-3}~\text{с}}{5 \cdot 10^5~\text{Ом}} = 10^4~\text{пФ} = 10\ 000~\text{пФ}
\end{equation}

\begin{table}[H] \label{tab:pr:dif}
	\centering
	\caption{Дифференцирование}
	\begin{tabular}{cccccc}
		\toprule
		$\nu,~\text{Гц}$ & $2A_x,~\text{В}$ & $2A_y,~\text{В}$ & $\left. 2U_y \right|_{U_x = 0},~\text{В}$ & $\sin \approx \frac{2U_y}{2A_y}$ & $K = \frac{2A_x}{2A_y}$ \\ \midrule
		100              & 15,70            & 0,10             & 0,10                                      & 1,000                            & 157,00                  \\
		500              & 19,60            & 0,58             & 0,58                                      & 1,000                            & 33,79                   \\
		700              & 14,20            & 0,60             & 0,60                                      & 1,000                            & 23,67                   \\
		800              & 14,00            & 0,68             & 0,68                                      & 1,000                            & 20,59                   \\
		900              & 11,00            & 0,61             & 0,61                                      & 1,000                            & 18,03                   \\
		1000             & 10,40            & 0,61             & 0,61                                      & 1,000                            & 17,05                   \\
		1500             & 7,20             & 0,64             & 0,64                                      & 1,000                            & 11,25                   \\
		2000             & 5,40             & 0,62             & 0,62                                      & 1,000                            & 8,71                    \\
		2500             & 8,20             & 1,22             & 1,19                                      & 0,975                            & 6,72                    \\
		3000             & 17,60            & 3,04             & 3,04                                      & 1,000                            & 5,79                    \\
		3500             & 30,00            & 6,04             & 5,84                                      & 0,967                            & 4,97                    \\
		4000             & 31,80            & 7,36             & 7,12                                      & 0,967                            & 4,32                    \\
		5000             & 31,80            & 9,00             & 8,40                                      & 0,933                            & 3,53                    \\
		6000             & 31,80            & 10,60            & 9,60                                      & 0,906                            & 3,00                    \\
		7000             & 31,80            & 12,20            & 10,80                                     & 0,885                            & 2,61                    \\
		8000             & 31,40            & 13,60            & 11,90                                     & 0,875                            & 2,31                    \\
		9000             & 31,40            & 14,60            & 12,20                                     & 0,836                            & 2,15                    \\
		10000            & 31,40            & 15,40            & 13,00                                     & 0,844                            & 2,04                    \\
		12000            & 31,40            & 17,20            & 13,80                                     & 0,802                            & 1,83                    \\
		14000            & 33,60            & 20,20            & 14,80                                     & 0,733                            & 1,66                    \\
		16000            & 27,40            & 17,60            & 12,20                                     & 0,693                            & 1,56                    \\
		18000            & 32,00            & 21,60            & 14,20                                     & 0,657                            & 1,48                    \\
		20000            & 31,00            & 22,00            & 13,40                                     & 0,609                            & 1,41                    \\
		30000            & 31,00            & 24,60            & 11,20                                     & 0,455                            & 1,26                    \\
		50000            & 30,40            & 26,20            & 7,80                                      & 0,298                            & 1,16                    \\
		100000           & 30,00            & 26,60            & 4,00                                      & 0,150                            & 1,13                   \\ \bottomrule
	\end{tabular}
\end{table}

\begin{table}[H] \label{tab:pr:int}
	\centering
	\caption{Интегрирование}
	\begin{tabular}{cccccc}
		\toprule
		$\nu,~\text{Гц}$ & $2A_x,~\text{В}$ & $2A_y,~\text{В}$ & $\left. 2U_y \right|_{U_x = 0},~\text{В}$ & $\sin \approx \frac{2U_y}{2A_y}$ & $K = \frac{2A_x}{2A_y}$ \\ \midrule
		20 & 19,8 & 11,80 & 2,8 & 0,237 & 0,60 \\ 
		40 & 20,0 & 10,00 & 5,6 & 0,560 & 0,50 \\ 
		60 & 19,6 & 8,20 & 6 & 0,732 & 	  0,42 \\ 
		80 & 19,6 & 6,88 & 5,84 & 0,849 & 0,35 \\ 
		100 & 19,8 & 5,84 & 5,28 & 0,904 & 0,29 \\ 
		120 & 20,2 & 4,96 & 4,48 & 0,903 & 0,25 \\ 
		150 & 20,2 & 4,00 & 3,68 & 0,920 & 0,20 \\ 
		200 & 20,2 & 3,04 & 2,88 & 0,947 & 0,15 \\ 
		300 & 20,2 & 2,08 & 1,92 & 0,923 & 0,10 \\ 
		400 & 20,2 & 1,54 & 1,5 & 0,974 & 0,08 \\ 
		500 & 20,2 & 1,26 & 1,22 & 0,968 & 0,06 \\ 
		1000 & 20,2 & 0,62 & 0,61 & 0,984 & 0,03 \\ 
		2000 & 20,2 & 0,31 & 0,31 & 1,000 & 0,02 \\ 
		3000 & 20,2 & 0,21 & 0,21 & 1,000 & 0,01 
		\\ \bottomrule
	\end{tabular}
\end{table}

\begin{figure}[H]
	\centering
	\includegraphics[width=1\linewidth]{graph/graph3}
	\caption{ Изменение коэффициента передачи $K$ при дифференцировании сигнала }
	\label{fig:pr:dif}
\end{figure}

\begin{figure}[H]
	\centering
	\includegraphics[width=1\linewidth]{graph/graph4}
	\caption{ Изменение коэффициента передачи $K$ при интегрировании сигнала }
	\label{fig:pr:int}
\end{figure}

Из таблиц \cref{tab:pr:dif} и \cref{tab:pr:int} видно, что при увеличении частоты входного сигнала коэффициент передачи падает

\subsection{Интегрирование и дифференцирование сигналов}
Рассмотрим 4 сигнала: урезанную синусоиду(\cref{fig:pr:theory}(1)), пилу(\cref{fig:pr:theory}(2)), треугольник(\cref{fig:pr:theory}(3)) и меандр(\cref{fig:pr:theory}(4)). \\ Графически продифференцируем(пункты $b$) и проинтегрируем сигналы(пункты $c$), чтобы оценить практический результат.

\begin{figure}[H]
	\centering
	\includegraphics[width=1\linewidth]{graph/charts/theory_charts_int_dif}
	\caption{}
	\label{fig:pr:theory}
\end{figure}

Соберём установку для интегрирования и дифференцирования. Будем считать измерения при наших параметрах (вычисленных ранее) $-$ основными. Наша задача изучить как будет вести себя сигнал при изменении этих параметров. 

\subsubsection{Изменение параметров интегрирования и дифференцирования}

Будем рассматривать следующие варианты:
\begin{enumerate}
	\item [-] Параметры выше основных по $R$
	\item [-] Параметры выше основных по $C$
	\item [-] Параметры выше основных по $R$ и $C$ 
	\item [-] Основные параметры
	\item [-] Параметры ниже основных по $R$
	\item [-] Параметры ниже основных по $C$
	\item [-] Параметры ниже основных по $R$ и $C$ 
\end{enumerate}
В случае малого разброса значений, приведём один график при максимальном отклонении от нормы по обоим параметрам. В противном случае будут приведены 3 графика: малое отклонение(1-2 шага), среднее(3-4 шага), максимальное отклонение.
Графики прилагаются отдельно.

\subsubsection{Изменение частоты сигнала}
Максимальная частота генератора равна $1~\text{кГц}$, поэтому будет изучать поведение выходного сигнала при уменьшении частоты входного сигнала. Графики прилагаются отдельно.


\subsection{Анализ результатов эксперимента}
Заметим, что оценка улучшения результатов затруднительна, так как на практике сигнал становиться шумным, что делает визуальный анализ неточным. При ухудшении параметров выходной сигнал начинает совпадать с входным, это отчётливо видно на примере: синусоиды, треугольника и пилы. В случае меандра нам  не удалось достичь совпадения из-за ограничения частоты генератора. 

Можно сделать вывод, что меандр является самым стабильным сигналом для операций дифференцирования и интегрирования, так как нам не удалось добиться полного искажения выходного сигнала, в отличии от других примеров.

\subsection{Разложение усечённой синусоиды в ряд Фурье}
Ряд Фурье позволяет заменить данную периодическую функцию бесконечной суммой гармонических функций (синусов и косинусов), каждые и которых имеют свои амплитуды и частоты. Если принять, что период функции $f(x)$ лежит в интервале от $L$ до $-L$, то ряд выглядит следующим образом:
\begin{align}
	 f(x) = \frac{a_0}2 + \sum\limits_{n = 1}^{\infty}\big( a_n \cos \big(\frac{n\pi x}{L}\big) + b_n \sin \big(\frac{n\pi x}{L}\big) \big) \text{, где:}
\end{align}

\[ a_0 = \frac1{L} \int_{-L}^{L} f(x) dx \]
\[ a_n = \frac1{L} \int_{-L}^{L} f(x) \cos \big(\frac{n\pi x}{L}\big) dx \]
\[ b_n = \frac1{L} \int_{-L}^{L} f(x) \sin \big(\frac{n\pi x}{L}\big) dx \]

По заданию будем считать, что существенными являются первые 7 гармоник спектра.