\section{Введение}
\subsection*{Цель} \addcontentsline{toc}{subsection}{Цель}
Исследовать устройство и принцип работы интегрирующих и дифференцирующих цепочек. Научиться работать с ними, провести операции с сигналами.

\subsection*{Задачи} \addcontentsline{toc}{subsection}{Задачи}
\begin{enumerate}
	\item Для дифференцирующего четырёхполюсника с постоянной времени \( \tau_0 = 10 \, \text{мкс} \) и интегрирующего четырёхполюсника с постоянной времени \( \tau_0 = 5 \, \text{мс} \) снимите зависимость модуля \( K(\omega) \) и аргумента \( \varphi(\omega) \) коэффициента передачи от частоты. Входной сигнал подавайте от звукового генератора, а величину выходного сигнала фиксируйте с помощью осциллографа, используя калиброванный коэффициент усиления. Сдвиг фаз определите по форме эллипса, которая получается на экране осциллографа при подаче входного сигнала на вертикальный (\(Y\)) канал, а выходного --- на горизонтальный (\(X\)) канал.
	
	Постройте графики зависимостей \( K(\omega) \) и \( \varphi(\omega) \). Попытайтесь качественно оценить, для какой области частот приближённо осуществляется дифференцирование и интегрирование.
	
	\item При помощи схемы, изображённой на рис. 3а, получите осциллограмму напряжения, представленную на рис. 3б, среднее значение которого равно нулю (т.е. равны площади, обозначенные штриховкой).
	\begin{figure}[!ht]
		\centering
		
		\begin{subfigure}[H]{0.49\linewidth}
			\centering
			\resizebox{0.9\linewidth}{!}{%
				\begin{circuitikz}
					\tikzstyle{every node}=[font=\small]
					\draw [ line width=0.6pt](2.75,13.75) to[D] (4.75,13.75);
					\draw [ line width=0.6pt](2.75,13.75) to[short, -o] (2.75,13.25) ;
					\draw [ line width=0.6pt](2.75,12) to[short, -o] (2.75,12.5) ;
					\draw [ line width=0.6pt](4.75,13.75) to[european resistor] (4.75,12);
					\draw [ line width=0.6pt](2.75,12) to[short] (5.75,12);
					\draw [line width=0.6pt](5.25,13) to[C] (6,13);
					\draw [line width=0.6pt, ->, >=Stealth] (5.25,13) -- (5,13);
					\draw [ line width=0.6pt](6,13) to[short, -o] (6.5,13) ;
					\draw [ line width=0.6pt](5.75,12) to[short, -o] (6.5,12) ;
					\node [font=\scriptsize] at (4.25,12.5) {$R$};
					\node [font=\scriptsize] at (3.75,13.25) {$D$};
					\node [font=\scriptsize] at (6,13.25) {$C$};
					\node [font=\scriptsize] at (6.75,12.5) {$U_\text{вх}(t)$};
					\node at (4.75,12) [circ] {};
				\end{circuitikz}
			}%
			\caption{Схема выпрямителя}
			\label{fig:6a}
		\end{subfigure}
		\hfill
		\begin{subfigure}[h]{0.49\linewidth}
			\centering
			\resizebox{0.9\linewidth}{!}{%
				\begin{circuitikz}
					\tikzstyle{every node}=[font=\footnotesize]
					\fill[blue!20] (3.08327,11.5) 
					-- plot[domain=3.08327:3.79726,samples=50,smooth] (\x,{0.85*sin(4.4*\x r-1 r ) +11.5})
					-- (3.79726,11.5) -- cycle;
					
					\fill[blue!20] (3.79726,11.5) 
					-- plot[domain=3.79726:3.94019,samples=50,smooth] (\x,{0.85*sin(4.4*\x r-1 r ) +11.5})
					-- (3.94019,11.5) -- cycle;
					
					\fill[blue!20] (4.36834,11.5) 
					-- plot[domain=4.36834:4.51126,samples=50,smooth] (\x,{0.85*sin(4.4*\x r-1 r ) +11.5})
					-- (4.51126,11.5) -- cycle;
					
					\fill[blue!20] (3.94019,11.5) 
					-- plot[domain=3.94019:4.36834,samples=50,smooth] (\x,{11})
					-- (4.36834,11.5) -- cycle;
					
					\draw[domain=2:2.51219,samples=50,smooth] plot (\x,{0.85*sin(4.4*\x r-1 r ) +11.5});
					\draw[domain=2.94034:3.94019,samples=50,smooth] plot (\x,{0.85*sin(4.4*\x r-1 r ) +11.5});
					\draw[domain=4.36834:5,samples=50,smooth] plot (\x,{0.85*sin(4.4*\x r-1 r ) +11.5});
					\draw [->, >=Stealth, line width=0.6pt] (2,10.5) -- (2,12.75);
					\draw [->, >=Stealth, line width=0.6pt] (1.75,11.5) -- (5.5,11.5);
					
					
					\draw (2.51219,11) to[short] (2.94034,11);
					\draw (3.94019,11) to[short] (4.36834,11);
					\node [font=\scriptsize] at (2.25,12.75) {$u$};
					\node [font=\scriptsize] at (5.25,11.75) {$2T$};
				\end{circuitikz}
			}%
			\caption{Осциллограмма напряжений}
			\label{fig:6b}
		\end{subfigure}
		\caption{ }
		\label{fig:6}
	\end{figure}
	
	Разложите эту функцию в ряд Фурье. Нарисуйте её амплитудный спектр. Считая существенными первые семь гармоник спектра, оцените параметры цепочек, пригодных для дифференцирования и интегрирования этой функции. Зарисуйте осциллограммы выходных напряжений для выбранных цепочек. По чертежу (рис. 3б) постройте производную и интеграл функции \( U_a(t) \) и сравните их с полученными осциллограммами.
	
	\item Подайте на вход осциллографа сигналы с генератора импульсов (меандр, треугольный и пилообразный). Зарисуйте их осциллограммы. Постройте графически производную и интеграл от этих сигналов.
	
	\item Подключив выход генератора импульсов ко входу четырёхполюсников, получите осциллограммы преобразованных сигналов. При неизменной частоте следования импульсов убедитесь, как влияет изменение постоянной времени на качество преобразования. Оцените постоянные времени, при которых, на ваш взгляд, наступает удовлетворительное дифференцирование и интегрирование. Сравните ваши оценки с теоретическими.
	
	\item Проделайте то же задание, изменяя частоту следования импульсов при неизменной \( \tau_0 \).
	
	При выполнении этих заданий зарисовывайте осциллограммы преобразования сигналов как в режиме, когда дифференцирование и интегрирование ещё не наступило, так и в режиме, когда, по-вашему, дифференцирование и интегрирование вполне удовлетворительно.
	
	\item Выясните, какой из трёх импульсных сигналов при прочих равных условиях <<легче>> дифференцируется или интегрируется. Обоснуйте ваш вывод со спектральной точки зрения.
\end{enumerate}



\subsection*{Приборы и оборудование} \addcontentsline{toc}{subsection}{Приборы и оборудование}
\begin{enumerate}
	\item Дифференцирующий и интегрирующий четырехполюсник;
	\item Функциональный генератор с модуляцией сигнала \texttt{AG1012F};
	\item Низкочастотный генератор \texttt{GW Instek GAG-810};
	\item Осциллограф цифровой \texttt{GDS-71022};
\end{enumerate}